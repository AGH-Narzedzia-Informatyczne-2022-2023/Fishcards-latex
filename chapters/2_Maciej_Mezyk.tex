\section{Maciej Mezyk}
Wyrażenie matematyczne: 
\[e^{i\pi}+1=0\]

\begin{figure} [htbp]
    \centering
    \includegraphics[scale=1.1]{pictures/euler.jpg}
    \caption{Leonhard Euler}
\end{figure}


Tabelka ~\ref{tab:tabela2}
\begin{table}[htbp]
    \centering
\begin{tabular}{lllll}
\hline
           & \textbf{a} & \textbf{b} & \textbf{c} & \textbf{d} \\ \hline
\textbf{1} & 1a         & 1b         & 1c         & 1d         \\
\textbf{2} & 2a         & 2b         & 2c         & 2d         \\
\textbf{3} & 3a         & 3b         & 3c         & 3d         \\ \hline
\end{tabular}
    \label{tab:tabela2}
\end{table}
\\

Lista numerowana:
\begin{enumerate}
    \item one
    \item two
    \item three
\end{enumerate}

 Lista nienumerowana:
\begin{itemize}
    \item one
    \item two
    \item three
\end{itemize}

Trees usually reproduce using seeds. Flowers and fruit may be present, but some trees, such as conifers, instead have pollen cones and seed cones. Palms, bananas, and bamboos also produce seeds, but tree ferns produce spores instead.

Trees play a significant role in reducing \textbf{erosion} and moderating the climate. They remove carbon dioxide from the atmosphere and store large quantities of carbon in their tissues. Trees and forests provide a habitat for many species of \underline{animals} and \emph{plants}. Tropical rainforests are among the most biodiverse habitats in the world.

Trunks occur both in "true" woody plants as well as non-woody plants such as palms and other monocots, though the internal physiology is different in each case. In all plants, trunks thicken over time due to the formation of secondary growth (or in monocots, pseudo-secondary growth). Trunks can be vulnerable to damage, including sunburn.